\documentclass[main.tex]{subfiles}

\begin{document}

\subsection{OpenLDAP}
OpenLDAP ist eine Implementierung des LDAP, die als freie Software unter der, BSD-Lizenz ähnlichen, 
OpenLDAP Public License veröffentlicht wird. OpenLDAP ist Bestandteil der meisten aktuellen
Linux-Distributionen und läuft auch unter verschiedenen Unix-Varianten, Mac OS X und verschiedenen Windows-Versionen.
Da OpenLDAP den LDAP-Standard verfolgt, ist es mit OpenLDAP möglich,
eine zentrale Benutzerdatenverwaltung aufzubauen und zentral zu warten.
\\\\
Kosten: Gratis (OpenLDAP Public License)
\\\\
\textbf{Vergleich zu anderen Lösungen}
Da OpenLDAP die Referenzimplementierung des Protokolls ist, werden Schemadateien sorgfältig
auf Protokollkonformität geprüft. Dies führt gelegentlich zu Fehlermeldungen, wenn mangelhafte Schemadateien,
die von Directory Server Agents (DSA) anderer Hersteller akzeptiert werden, in ein OpenLDAP System übertragen werden.

Durch die Bereitstellung unterschiedlicher Backends und Overlays lassen sich Protokollerweiterungen und
erweiterte Operationen (extended Operations) sehr leicht realisieren. Das SQL Backend leitet die
Suchergebnisse einer RDBM-Suche an den DSA weiter, so dass der auftraggebende LDAP Client ein
protokollgerechtes Datenpaket empfängt.
 \\\\
 Active Directory scheidet aus da es nur für Windows verfügbar ist.
 
 \subsection{Webserver}
 Als Webserver wird nginx mit PHP-FPM auf Grund der Performance gegenüber Apache empfohlen
 
 \subsection{Mailserver}
 Als Mailserver wird eine Kombination aus Postfix und Dovecot verwendet.
 
 \subsection{Fileserver}
 TODO

\end{document}
