\documentclass[main.tex]{subfiles}

\begin{document}

\subsection{Ausgangssituation}
Sie sind als Netzwerk- und Systemadministrator für bei einem Startup-Unternehmen angestellt worden. Dieses benötigt eine Vielzahl von Netzwerk-Services (Mail, Web, Filesharing, Drucken, Login am Desktop, ...) für die eine zentrale Benutzerverwaltung eingeführt werden soll.\\\\
Die Mitarbeiter des Unternehmens arbeiten mit heterogenen Systemen - manche unter Linux, machen unter Windows, manche unter Mac OS und einige mit FreeBSD. Die zentrale Authentifizierung soll bei all diesen Systemen funktionieren. Als nice-to-have Feature wünscht sich das Unternehmen eine Single-Sing-On-Lösung.

\subsection{Aufgabenstellung}
Vergleichen Sie für dieses Unternehmen die Kosten (Hardware-Anforderungen, Lizenzkosten, geschätzter Administrationsaufwand/Monat) für die Einführung einer zentralen Benutzerverwaltung (OpenLDAP oder Active Directory) sowie der dazugehörigen Server-Systeme (Webserver, Mailserver, Fileserver).\\\\
Überlegen Sie auch, ob für die gegebene Aufgabenstellung eine lokale Server-Infrastruktur benötigt wird, oder ob manche/alle Dienste in ein Cloud-Service ausgelagert werden können (z.B. AWS, Azure, Google Cloud, ...).

\subsection{Abgabe}
Geben Sie die fertige Kosten- und Aufwandsabschätzung als ordentlich formatiertes Dokument im PDF Format ab.
\\\\
Das Dokument muss (zusätzlich zum "ordentlichen Aufbau", also Titelseite, Inhaltsverzeichnis, Kopf- und Fußzeilen, ...) die Evaluation der zur Auswahl stehenden Software-Lösungen sowie eine detaillierte Zeit- und Kostenaufstellung enthalten.
\\\\
Die Arbeit in Gruppen (maximal 3 Personen, andere Gruppenzusammenstellung als bei der letzten Übung!) ist erlaubt.
\end{document}
